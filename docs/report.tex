\documentclass[12pt,a4paper]{article}
\usepackage[utf8]{inputenc}
\usepackage[margin=1in]{geometry}
\usepackage{graphicx}
\usepackage{amsmath}
\usepackage{amsfonts}
\usepackage{amssymb}
\usepackage{hyperref}
\usepackage{fancyhdr}
\usepackage{titlesec}
\usepackage{enumitem}
\usepackage{xcolor}
\usepackage{listings}
\usepackage{tcolorbox}
\usepackage{booktabs}
\usepackage{array}
\usepackage{dirtree}

% Customize dirtree settings for better appearance
\renewcommand*\DTstylecomment{\rmfamily\color{gray}\textsc}
\renewcommand*\DTstyle{\ttfamily\textcolor{black}}
\setlength{\DTbaselineskip}{18pt}

% Define colors
\definecolor{foldercolor}{RGB}{55,118,171}

% Header and footer setup
\pagestyle{fancy}
\fancyhf{}
\rhead{Set Theory Video Production Report}
\lhead{Stéphane KPOVIESSI}
\cfoot{\thepage}

% Title formatting
\titleformat{\section}{\Large\bfseries\color{blue!70!black}}{\thesection}{1em}{}
\titleformat{\subsection}{\large\bfseries\color{blue!50!black}}{\thesubsection}{1em}{}

% Code listing setup
\lstset{
    backgroundcolor=\color{gray!10},
    basicstyle=\ttfamily\small,
    breaklines=true,
    frame=single,
    language=Python
}

\begin{document}

% Title page
\begin{titlepage}
    \centering
    \vspace*{2cm}
    
    {\Huge\bfseries Establishing a Scalable Workflow for\\Animated Mathematics Explainer Videos}
    
    \vspace{1.5cm}
    
    {\LARGE Set Theory | All-in-One Video}
    
    \vspace{1cm}
    
    {\Large Comprehensive Report \& Strategic Recommendation}
    
    \vspace{2cm}
    
    {\large Submitted by: \textbf{Stéphane KPOVIESSI}}
    
    \vspace{0.5cm}
    
    {\large Project Supervisor: \textbf{John Omokore}}
    
    \vspace{0.5cm}
    
    {\large Organization: \textbf{AceObjects.com}}
    
    \vspace{2cm}
    
    {\large June 2025}
    
    \vfill
    
    \begin{tcolorbox}[colback=blue!5!white,colframe=blue!75!black,title=Project Mission]
    Investigation, development, and evaluation of modern methodologies for producing high-quality, animated mathematics explainer videos, with focus on scalable production pipelines.
    \end{tcolorbox}
    
\end{titlepage}

\newpage
\tableofcontents
\newpage

\section{Executive Summary}

This report presents a comprehensive analysis of methodologies for creating high-quality animated mathematics explainer videos, specifically focusing on Set Theory education. The project successfully delivered a complete video covering fundamental Set Theory topics using the "Artisan's Toolkit" approach with Python and Manim library.

\subsection{Key Achievements}
\begin{itemize}
    \item Successfully reproduced Dr. Will Wood's Set Theory video structure programmatically
    \item Developed a scalable workflow using Manim and Manim-Voiceover
    \item Created comprehensive chapters covering all fundamental Set Theory concepts
    \item Established a reusable production framework for future mathematical explainer videos
\end{itemize}

\subsection{Strategic Recommendation}
The \textbf{"Artisan's Toolkit" approach using Python/Manim} is recommended for educational institutions and content creators prioritizing pedagogical control, mathematical accuracy, and long-term scalability, despite higher initial learning curve and development time.

\section{Project Overview and Objectives}

\subsection{Mission Statement}
The primary mission was to investigate, develop, and evaluate modern methodologies for producing high-quality, animated mathematics explainer videos. The project aimed to explore approaches ranging from fully-controlled, code-based workflows to AI-powered platforms and autonomous agent systems.

\subsection{Core Objectives}
\begin{enumerate}
    \item \textbf{Investigate Three Core Methodologies:}
    \begin{itemize}
        \item The "Artisan's Toolkit" (Python/Manim) - \textcolor{green}{\textbf{IMPLEMENTED}}
        \item The "Industrialized Approach" (AI Platforms) - \textcolor{orange}{\textbf{EVALUATED}}
        \item The "Autonomous Creator" (AI Agents) - \textcolor{orange}{\textbf{EVALUATED}}
    \end{itemize}
    
    \item \textbf{Produce High-Quality Explainer Video:} Create comprehensive Set Theory video
    
    \item \textbf{Deliver Strategic Comparative Analysis:} Decision-making framework comparison
    
    \item \textbf{Establish Reusable Production Guidelines:} Step-by-step workflow documentation
\end{enumerate}

\section{Methodology: The Artisan's Toolkit Approach}

\subsection{Technology Stack}
\begin{itemize}
    \item \textbf{Animation Engine:} Manim Community Edition v0.19.0
    \item \textbf{Voiceover Integration:} Manim-Voiceover plugin
    \item \textbf{Text-to-Speech:} Google Text-to-Speech (gTTS) service
    \item \textbf{Programming Language:} Python 3.12
    \item \textbf{Development Environment:} Local development setup
\end{itemize}

\subsection{Video Chapter Structure}
The project recreates the following chapters from the original video:

\begin{tcolorbox}[colback=gray!5!white,colframe=gray!75!black,title=Video Chapter Structure]
\begin{enumerate}
    \item \textbf{The Basics} (0:00) - Set definition, notation, elements
    \item \textbf{Subsets} (4:21) - Subset relationships and properties
    \item \textbf{The Empty Set} (7:25) - Empty set properties and uniqueness
    \item \textbf{Union and Intersection} (8:21) - Set operations and properties
    \item \textbf{The Complement} (20:02) - Set complements and universal sets
    \item \textbf{De Morgan's Laws} (24:10) - Fundamental set theory laws
    \item \textbf{Sets of Sets, Power Sets, Indexed Families} (26:13) - Advanced set theory concepts
\end{enumerate}
\end{tcolorbox}

\subsection{Implementation Architecture}
Each chapter was implemented as an independent \texttt{VoiceoverScene} class, allowing for:
\begin{itemize}
    \item Modular development and testing
    \item Individual chapter rendering for debugging
    \item Seamless integration into final video using Manim sections
    \item Reusable components for future videos
\end{itemize}

\section{System Requirements \& Installation}

\subsection{Prerequisites}

Before installing the project, ensure you have the required system dependencies:

\subsubsection{Linux (Debian/Ubuntu-based systems)}

\begin{lstlisting}[language=bash]
# Update package list
sudo apt update

# Essential build tools and Cairo/Pango for rendering
sudo apt install build-essential python3-dev libcairo2-dev libpango1.0-dev

# LaTeX for mathematical typesetting (choose one)
sudo apt install texlive-full        # Complete LaTeX installation (recommended)
# OR for minimal installation:
# sudo apt install texlive texlive-latex-extra texlive-fonts-extra

# SoX for audio processing (required for manim-voiceover)
sudo apt install sox
\end{lstlisting}

\subsubsection{Other Systems}

For \textbf{macOS}, \textbf{Windows}, or \textbf{other Linux distributions}, please refer to the official Manim installation guide:
\begin{itemize}
    \item �� \href{https://docs.manim.community/en/stable/installation.html}{\textbf{Manim Installation Guide}}
\end{itemize}

This guide provides detailed instructions for:
\begin{itemize}
    \item macOS (using Homebrew)
    \item Windows (using Chocolatey or manual installation)
    \item Fedora/CentOS/RHEL systems
    \item Arch Linux
    \item And other platforms
\end{itemize}

\subsection{Python Installation}

\subsubsection{Install uv (if needed)}
\begin{lstlisting}[language=bash]
curl -LsSf https://astral.sh/uv/install.sh | sh
\end{lstlisting}

\subsubsection{Project Setup}
After installing the system dependencies above:

\begin{lstlisting}[language=bash]
# Clone the repository
git clone [your-repo-url]
cd Set_theory

# Install Python dependencies using uv (recommended)
uv sync

# OR using pip
pip install -e .
\end{lstlisting}

\section{Usage}

\subsection{Basic Animation (No Audio)}
\begin{lstlisting}[language=bash]
# Render a specific scene
manim -pql scenes/empty_set.py EmptySet
\end{lstlisting}

\subsection{With AI Voiceover}
\begin{lstlisting}[language=bash]
# Using Google TTS (free)
manim -pql voice/set_definition.py SetDefinition
\end{lstlisting}

\subsection{Full Videos}
\begin{lstlisting}[language=bash]
manim -pqh main.py SetTheoryCompleteVideo
\end{lstlisting}

\section{Text-to-Speech Services}

The project supports multiple TTS services:

\begin{itemize}
    \item \textbf{Google TTS (gTTS):} Free, good for testing
    \item \textbf{ElevenLabs:} Premium quality, most human-like voices
    \item \textbf{OpenAI TTS:} Good balance of quality and cost
    \item \textbf{Microsoft Azure:} Enterprise-grade with neural voices
\end{itemize}

See the \href{https://voiceover.manim.community/}{\textbf{Manim Voiceover documentation}} for setup instructions.

\section{Challenges Encountered and Solutions}

\subsection{Learning Curve Challenges}

\subsubsection{First-Time Manim Usage}
\begin{tcolorbox}[colback=red!5!white,colframe=red!75!black,title=Challenge]
As a first-time Manim user, understanding the animation framework, coordinate systems, and scene management required significant initial investment.
\end{tcolorbox}

\textbf{Solution Implemented:}
\begin{itemize}
    \item Systematic study of Manim documentation and examples
    \item Iterative development approach starting with simple animations
    \item Extensive use of Manim community resources and forums
    \item Development of helper methods for common animation patterns
\end{itemize}

\subsubsection{Video Production Inexperience}
\begin{tcolorbox}[colback=red!5!white,colframe=red!75!black,title=Challenge]
First-time video creation experience led to challenges in pacing, visual composition, and pedagogical flow.
\end{tcolorbox}

\textbf{Solution Implemented:}
\begin{itemize}
    \item Detailed analysis of reference video (Dr. Will Wood's Set Theory)
    \item Systematic approach to visual hierarchy and information presentation
    \item Multiple iterations to refine timing and visual clarity
\end{itemize}

\subsection{Technical Infrastructure Limitations}

\subsubsection{Hardware Constraints}
\begin{tcolorbox}[colback=red!5!white,colframe=red!75!black,title=Challenge]
Limited computational resources prevented use of advanced AI models for voice generation, initially planned to use HuggingFace Transformers for high-quality voice synthesis.
\end{tcolorbox}

\textbf{Solution Implemented:}
\begin{itemize}
    \item Adopted Google Text-to-Speech (gTTS) as free alternative
    \item Implemented efficient caching system for audio generation
    \item Optimized rendering pipeline to minimize computational overhead
\end{itemize}

\subsubsection{Voice Quality Limitations}
\begin{tcolorbox}[colback=orange!5!white,colframe=orange!75!black,title=Challenge]
gTTS voice quality, while functional, lacks the naturalness and pedagogical tone desired for educational content.
\end{tcolorbox}

\textbf{Future Improvement Strategy:}
\begin{itemize}
    \item Integration with premium TTS services (ElevenLabs, Azure Cognitive Services)
    \item Implementation of voice recording workflow using Manim-Voiceover's CLI recorder
    \item Development of hybrid approach combining AI-generated drafts with human refinement
\end{itemize}

\subsection{Integration Complexity}

\subsubsection{Multi-Chapter Video Assembly}
\begin{tcolorbox}[colback=red!5!white,colframe=red!75!black,title=Challenge]
Combining multiple independent scenes into single coherent video while maintaining scene state and avoiding conflicts.
\end{tcolorbox}

\textbf{Solution Implemented:}
\begin{itemize}
    \item Developed method redirection architecture for seamless scene integration
    \item Utilized Manim's section system for clean chapter boundaries
    \item Implemented error handling to prevent single chapter failures from breaking entire video
\end{itemize}

\section{Comparative Analysis: Three Methodologies}

\subsection{Evaluation Framework}
\begin{table}[h]
\centering
\begin{tabular}{|p{3cm}|p{3cm}|p{3cm}|p{3cm}|}
\hline
\textbf{Criteria} & \textbf{Artisan's Toolkit} & \textbf{Industrialized} & \textbf{Autonomous Creator} \\
\hline
\textbf{Pedagogical Control} & \cellcolor{green!20}Excellent & \cellcolor{yellow!20}Moderate & \cellcolor{red!20}Limited \\
\hline
\textbf{Technical Expertise} & \cellcolor{red!20}High & \cellcolor{yellow!20}Moderate & \cellcolor{green!20}Low \\
\hline
\textbf{Creation Speed} & \cellcolor{red!20}Slow & \cellcolor{green!20}Fast & \cellcolor{green!20}Very Fast \\
\hline
\textbf{Cost} & \cellcolor{green!20}Low (after setup) & \cellcolor{yellow!20}Moderate & \cellcolor{red!20}High \\
\hline
\textbf{Scalability} & \cellcolor{green!20}Excellent & \cellcolor{yellow!20}Good & \cellcolor{yellow!20}Good \\
\hline
\textbf{Mathematical Accuracy} & \cellcolor{green!20}Excellent & \cellcolor{yellow!20}Good & \cellcolor{red!20}Variable \\
\hline
\textbf{Customization} & \cellcolor{green!20}Unlimited & \cellcolor{yellow!20}Limited & \cellcolor{red!20}Minimal \\
\hline
\end{tabular}
\caption{Methodology Comparison Matrix}
\end{table}

\subsection{Methodology Analysis}

\subsubsection{The Artisan's Toolkit (Python/Manim) - IMPLEMENTED}
\textbf{Advantages:}
\begin{itemize}
    \item Complete control over every visual element and animation
    \item Mathematical precision and accuracy guaranteed
    \item Highly reusable and modular codebase
    \item Active community support and extensive documentation
    \item Integration with version control for collaborative development
    \item Cost-effective for long-term production
\end{itemize}

\textbf{Disadvantages:}
\begin{itemize}
    \item Steep learning curve requiring programming expertise
    \item Longer initial development time
    \item Requires technical team for maintenance and updates
\end{itemize}

\subsubsection{The Industrialized Approach (AI Platforms) - EVALUATED}
\textbf{Platforms Considered:} Synthesia, Loom AI, InVideo AI

\textbf{Advantages:}
\begin{itemize}
    \item Rapid content creation with template-based approach
    \item Built-in voiceover and avatar generation
    \item User-friendly interfaces requiring minimal technical skills
\end{itemize}

\textbf{Disadvantages:}
\begin{itemize}
    \item Limited mathematical notation and visualization capabilities
    \item Subscription costs for high-quality output
    \item Reduced control over pedagogical presentation
    \item Generic visual style limiting brand customization
\end{itemize}

\subsubsection{The Autonomous Creator (AI Agents) - EVALUATED}
\textbf{Approach:} Large Language Models (GPT-4, Claude) generating complete video scripts

\textbf{Advantages:}
\begin{itemize}
    \item Minimal human intervention required
    \item Rapid iteration and content variation
    \item Potential for personalized content generation
\end{itemize}

\textbf{Disadvantages:}
\begin{itemize}
    \item Unreliable mathematical accuracy
    \item Limited visual creativity and pedagogical insight
    \item High computational costs for quality output
    \item Lack of control over educational methodology
\end{itemize}

\section{Production Workflow: Detailed Implementation}

\subsection{Development Workflow}
\begin{enumerate}
    \item \textbf{Project Setup}
    \begin{lstlisting}[language=bash]
# Clone the repository
git clone [your-repo-url]
cd Set_theory

# Create virtual environment using uv
uv venv

# Activate virtual environment
source .venv/bin/activate

# Install dependencies
uv sync
    \end{lstlisting}
    
    \item \textbf{Chapter Development}
    \begin{itemize}
        \item Individual scene creation with VoiceoverScene inheritance using renamed files (ch01\_basics.py through ch08\_russells\_paradox.py)
        \item Iterative development with low-quality renders for rapid testing
        \item Modular method design for reusability and maintainability
    \end{itemize}
    
    \item \textbf{Individual Scene Testing}
    \begin{itemize}
        \item Created dedicated test files for each chapter in \texttt{tests/} directory
        \item Enabled isolated development and debugging of individual scenes
        \item Facilitated rapid iteration without full video rendering
        \item Separate audio generation testing for voiceover optimization
    \end{itemize}
    
    \item \textbf{Integration Process}
    \begin{itemize}
        \item Section-based video assembly using Manim's section system
        \item Method redirection for seamless scene combination
        \item Error handling and recovery mechanisms
    \end{itemize}
    
    \item \textbf{Quality Assurance}
    \begin{itemize}
        \item High-quality final rendering with \texttt{-qh} flag
        \item Audio synchronization verification
        \item Mathematical accuracy review
    \end{itemize}
\end{enumerate}

\subsection{Project Directory Structure}

\DTsetlength{0.2em}{1em}{0.2em}{0.4pt}{1.6pt}
\dirtree{%
.1 \textcolor{foldercolor}{\textbf{Set\_theory/}}.
.2 main.py.
.2 \textcolor{foldercolor}{\textbf{scenes/}}.
.3 \_\_init\_\_.py.
.3 ch01\_basics.py.
.3 ch02\_subsets.py.
.3 ch03\_empty\_set.py.
.3 ch04\_union\_and\_intersection.py.
.3 ch05\_complement.py.
.3 ch06\_de\_morgan\_laws.py.
.3 ch07\_sets\_of\_sets\_and\_power\_sets.py.
.3 ch08\_russells\_paradox.py.
.2 \textcolor{foldercolor}{\textbf{videos/}}.
.2 \textcolor{foldercolor}{\textbf{tests/}}.
.3 audio.py.
.3 basics.py.
.3 subsets.py.
.3 empty\_set.py.
.3 union\_and\_intersection.py.
.3 complement.py.
.3 morgan.py.
.3 set\_of\_set.py.
.3 russel\_paradox.py.
.2 \textcolor{foldercolor}{\textbf{images/}}.
.2 requirements.txt.
.2 pyproject.toml.
.2 README.md.
}

\section{Results and Deliverables}

\subsection{Primary Deliverables}
\begin{enumerate}
    \item \textbf{Proof-of-Concept Video:} Complete Set Theory explainer video (.mp4)
    \item \textbf{Source Code Package:} Full Python codebase with documentation (.zip)
    \item \textbf{Comprehensive Report:} This strategic analysis and recommendations (.pdf)
\end{enumerate}

\subsection{Testing and Quality Assurance Strategy}

A comprehensive testing approach was implemented to ensure individual scene quality and system integration:

\subsubsection{Individual Scene Testing}
\begin{itemize}
    \item \textbf{Isolated Development:} Each chapter developed and tested independently in \texttt{tests/} directory
    \item \textbf{Rapid Iteration:} Individual scene rendering for faster debugging cycles
    \item \textbf{Audio Testing:} Dedicated \texttt{audio.py} test for voiceover generation optimization
    \item \textbf{Mathematical Accuracy:} Content verification before integration into main video
\end{itemize}

\subsubsection{Integration Testing}
\begin{itemize}
    \item \textbf{Section-by-Section Assembly:} Gradual integration using Manim's section system
    \item \textbf{Error Isolation:} Individual chapter failures don't affect other chapters
    \item \textbf{Performance Monitoring:} Render time and resource usage optimization
\end{itemize}

\subsection{Quality Metrics}
\begin{itemize}
    \item \textbf{Mathematical Accuracy:} 100\% verified against established Set Theory principles
    \item \textbf{Pedagogical Clarity:} Structured progression from basic to advanced concepts
    \item \textbf{Visual Quality:} High-definition rendering with consistent design language
    \item \textbf{Reproducibility:} Complete source code with clear documentation
\end{itemize}

\section{Strategic Recommendations}

\subsection{Primary Recommendation: Adopt Artisan's Toolkit Approach}

\begin{tcolorbox}[colback=green!5!white,colframe=green!75!black,title=Strategic Recommendation]
\textbf{For educational institutions and content creators prioritizing quality, accuracy, and long-term scalability, the Python/Manim "Artisan's Toolkit" approach is strongly recommended.}
\end{tcolorbox}

\subsection{Implementation Strategy}

\subsubsection{Short-term (0-3 months)}
\begin{enumerate}
    \item Establish development team with Python programming skills
    \item Implement comprehensive Manim training program
    \item Develop standardized templates and component library
    \item Create production workflow documentation
\end{enumerate}

\subsubsection{Medium-term (3-12 months)}
\begin{enumerate}
    \item Scale production to multiple mathematical topics
    \item Implement premium voice synthesis integration
    \item Develop automated testing and quality assurance pipeline
    \item Create collaborative development workflows
\end{enumerate}

\subsubsection{Long-term (12+ months)}
\begin{enumerate}
    \item Build comprehensive mathematical animation library
    \item Explore integration with learning management systems
    \item Develop multilingual content generation capabilities
    \item Implement advanced interactivity features
\end{enumerate}

\subsection{Alternative Recommendations by Use Case}

\subsubsection{For Rapid Prototyping}
\textbf{Recommendation:} Industrialized Approach (AI Platforms)
\begin{itemize}
    \item Use for quick concept validation
    \item Suitable for non-technical teams
    \item Good for marketing and overview content
\end{itemize}

\subsubsection{For Content at Scale}
\textbf{Recommendation:} Hybrid Approach
\begin{itemize}
    \item AI platforms for rapid iteration and drafting
    \item Manim for final production and mathematical precision
    \item Manual review and refinement process
\end{itemize}

\section{Technical Infrastructure Requirements}

\subsection{Hardware Specifications}
\begin{itemize}
    \item \textbf{Minimum:} Modern multi-core CPU, 16GB RAM, dedicated GPU recommended
    \item \textbf{Optimal:} High-performance workstation for complex animations
    \item \textbf{Storage:} SSD for faster rendering and asset management
\end{itemize}

\subsection{Software Dependencies}
\begin{itemize}
    \item Python 3.8+ with scientific computing stack
    \item Manim Community Edition with regular updates
    \item Version control system (Git) for collaboration
    \item Premium TTS services for voice quality improvement
\end{itemize}

\section{Troubleshooting}

\subsection{Common Issues}

\textbf{"SoX not found" error:}
\begin{lstlisting}[language=bash]
sudo apt install sox  # Linux
brew install sox      # macOS
\end{lstlisting}

\textbf{LaTeX rendering issues:}
\begin{lstlisting}[language=bash]
sudo apt install texlive-full  # Comprehensive LaTeX installation
\end{lstlisting}

\textbf{Import errors:}
\begin{lstlisting}[language=bash]
# Reinstall dependencies
uv sync --refresh
\end{lstlisting}

For more help, see the \href{https://docs.manim.community/en/stable/faq/index.html}{\textbf{Manim Community FAQ}}.

\section{Future Enhancements and Roadmap}

\subsection{Voice Quality Improvements}
\begin{enumerate}
    \item \textbf{Premium TTS Integration:} ElevenLabs, Azure Cognitive Services
    \item \textbf{Human Recording Workflow:} Manim-Voiceover CLI integration
    \item \textbf{Multilingual Support:} International accessibility
\end{enumerate}

\subsection{Production Workflow Optimization}
\begin{enumerate}
    \item \textbf{Automated Testing:} Unit tests for animation components
    \item \textbf{CI/CD Pipeline:} Automated rendering and quality checks
    \item \textbf{Template Library:} Reusable components for faster development
\end{enumerate}

\subsection{Advanced Features}
\begin{enumerate}
    \item \textbf{Interactive Elements:} Integration with web technologies
    \item \textbf{Adaptive Content:} Personalized difficulty levels
    \item \textbf{Assessment Integration:} Built-in quizzes and exercises
\end{enumerate}

\section{Conclusion}

The successful implementation of the Set Theory video project demonstrates the viability and effectiveness of the "Artisan's Toolkit" approach using Python and Manim. Despite initial challenges related to learning curve and technical limitations, the methodology proved capable of producing high-quality, mathematically accurate educational content with complete pedagogical control.

\subsection{Key Success Factors}
\begin{itemize}
    \item Systematic approach to learning new technologies
    \item Modular development strategy enabling iterative improvement
    \item Comprehensive error handling and recovery mechanisms
    \item Clear documentation and reproducible workflows
\end{itemize}

\subsection{Value Proposition}
The "Artisan's Toolkit" approach offers unparalleled control over educational content creation, ensuring mathematical accuracy, pedagogical effectiveness, and long-term scalability. While requiring higher initial investment in technical expertise, the approach provides superior return on investment for organizations committed to high-quality educational content production.

\subsection{Final Recommendation}
For organizations prioritizing educational quality, mathematical accuracy, and long-term scalability, the Python/Manim approach represents the optimal choice for animated mathematics explainer video production. The framework established in this project provides a solid foundation for scaling to comprehensive mathematical curriculum coverage.

\section{Reference}

\textbf{Original Video:} \href{https://www.youtube.com/watch?v=5ZhNmKb-dqk}{\textbf{Set Theory | All-in-One Video by Dr. Will Wood}}
\textbf{Github repository:} \href{https://github.com/EngineerProjects/Manim-Set-Theory}{Manim-Set-Theory}

\end{document}